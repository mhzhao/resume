%%%%%%%%%%%%%%%%%
% This is an example CV created using mhcv.cls (v1.0, 10 Feb 2021) adapted by
% Minghui Zhao(zhaominghui2011@gmail.com) from AltaCV 
%
%% It may be distributed and/or modified under the
%% conditions of the LaTeX Project Public License, either version 1.3
%% of this license or (at your option) any later version.
%% The latest version of this license is in
%%    http://www.latex-project.org/lppl.txt
%% and version 1.3 or later is part of all distributions of LaTeX
%% version 2003/12/01 or later.
%%%%%%%%%%%%%%%%

%% If you are using \orcid or academicons
%% icons, make sure you have the academicons
%% option here, and compile with XeLaTeX
%% or LuaLaTeX.
% \documentclass[10pt,a4paper,academicons]{mhcv}

%% Use the "normalphoto" option if you want a normal photo instead of cropped to a circle
% \documentclass[10pt,a4paper,normalphoto]{mhcv}

\documentclass[10pt,a4paper,ragged2e]{mhcv}

%% mhcv uses the fontawesome and academicon fonts
%% and packages.
%% See texdoc.net/pkg/fontawecome and http://texdoc.net/pkg/academicons for full list of symbols. You MUST compile with XeLaTeX or LuaLaTeX if you want to use academicons.

% Change the page layout if you need to
\geometry{left=1cm,right=9cm,marginparwidth=6.8cm,marginparsep=1.2cm,top=1.25cm,bottom=1.25cm}
\usepackage{hyperref}
\hypersetup{
  colorlinks=true,
  linkcolor=blue,
  filecolor=magenta,
  urlcolor=cyan,
}
% Change the font if you want to, depending on whether
% you're using pdflatex or xelatex/lualatex
\ifxetexorluatex
  % If using xelatex or lualatex:
  \usepackage{fontspec}
  \setmainfont{Carlito}[
    Extension = .ttf,
    UprightFont = *-Regular,
    BoldFont = *-Bold,
    ItalicFont = *-Italic,
    BoldItalicFont = *-BoldItalic
  ]
\else
  % If using pdflatex:
  \usepackage[utf8]{inputenc}
  \usepackage[T1]{fontenc}
  \usepackage[default]{lato}
\fi

% Change the colours if you want to
\definecolor{VividPurple}{HTML}{3E0097}
\definecolor{SlateGrey}{HTML}{2E2E2E}
\definecolor{LightGrey}{HTML}{37474F}
%\colorlet{heading}{VividPurple}
%\colorlet{accent}{VividPurple}
\colorlet{emphasis}{SlateGrey}
\colorlet{body}{LightGrey}

% Change the bullets for itemize and rating marker
% for \cvskill if you want to
\renewcommand{\itemmarker}{{\small\faIcon[regular]{bell}}}
\renewcommand{\ratingmarker}{\faIcon[regular]{circle}}

%% sample.bib contains your publications
\addbibresource{sample.bib}

\begin{document}
\name{MINGHUI ZHAO \textcolor{gray}{| R\'esum\'e}}
\tagline{Seeking 2021 summer internship @Data Science @Machine Learning @Artificial Intelligence}
% Cropped to square from https://en.wikipedia.org/wiki/Marissa_Mayer#/media/File:Marissa_Mayer_May_2014_(cropped).jpg, CC-BY 2.0
%\photo{2.5cm}{American-passport-photo}
%\photo{2.5cm}{little_boy}
\personalinfo{%
  % Not all of these are required!
  % You can add your own with \printinfo{symbol}{detail}
    \email{mhzhao@iastate.com}
    \phone{+01 515-598-6213}
    \location{Ames, Iowa, US}
    \linkedin{linkedin.com/in/mhzhao} \\
    \github{github.com/mhzhao} 
    \kaggle{kaggle.com/youtianren} 
%   \orcid{orcid.org/0000-0000-0000-0000} % Obviously making this up too. If you want to use this field (and also other academicons symbols), add "academicons" option to \documentclass{mhcv}
}

%% Make the header extend all the way to the right, if you want.
\begin{fullwidth}
\makecvheader
\end{fullwidth}

%% Depending on your tastes, you may want to make fonts of itemize environments slightly smaller
\AtBeginEnvironment{itemize}{\small}

%% Provide the file name containing the sidebar contents as an optional parameter to \cvsection.
%% You can always just use \marginpar{...} if you do
%% not need to align the top of the contents to any
%% \cvsection title in the "main" bar.

\cvsection[page1sidebar]{Research \& Work Experience}

\iftrue
\cvevent{Physics analysis -- Data calibration and analysis}{Leader}{Aug 2017 -- Current}{Ames, IA, U.S}
\begin{itemize}
  \item Monte Carlo simulation of proton-proton collision and Drell Yan $Z^0/\gamma^*\rightarrow e^-/e^+$ analysis, extract $e^-/e^+$ signal pair from background which is $10^6$ larger than the signal. By using \textbf{machine learning method of boost decision tree}, we can improve signal/background ratio in 2 order ($10^2$).
    \item Experimental data calibration and analysis, using gaussian and polynomial fitting for calibration.
    \item Develop c++ classes for data calibration and analysis, and a lot of effort for plotting and data visualization. 
\end{itemize}
\cvtag{Shell scripting} 
\cvtag{C/C++}
\cvtag{R/Python}
\cvtag{Perl}
\cvtag{Machine Learning}

\vspace{10pt}

\cvevent{Experiment -- Detector assemble and maintnance}{Detector expert \hfill Brookhaven National Laboratory}{May 2016 -- Aug 2017}{Upton, NY, U.S}
\begin{itemize}
    \item Assembled our scintillator detector which called FPOST, tested and calibrated sillicon photomultiplier (SiPM) device. 
    \item I was the FPOST detector maintnance expert during the experiment. 
\end{itemize}
\cvtag{Oscilloscope} 
\cvtag{Electronics}
\cvtag{Electric circuit}
\cvtag{C/C++}
\cvtag{Shell scripting}

\vspace{10pt}

\cvevent{Model Simulation and detector operation}{Calculation and experimental skills \hfill }{Aug 2011 -- Jul 2014}{Beijing \& Lanzhou, China}
\begin{itemize}
  \item Model simulation and experimental data analysis of ${}^{40}Ca + {}^{40}Ca$ and ${}^{129}Xe+{}^{120}Sn$ collision, a paper pusblished. 
    \item Detecting cosmic ray muons to abtain experimental skills.
\end{itemize}
\cvtag{C/C++} 
\cvtag{Shell scripting}
\cvtag{Oscilloscope}
\cvtag{Electric circuit}

\vspace{10pt}

\cvsection{Education}

\cvevent{Ph.D, majoring in nuclear physics}{\faIcon{graduation-cap} Iowa State University}{Aug 2014 -- Current}{Ames, Iowa, U.S}
\cvevent{MSc, majoring in nuclear physics}{\faIcon{graduation-cap} Institute of Modern Physics, UCAS}{Sept 2012 -- Aug 2014}{Lanzhou, China}
\cvevent{MSc, first year, major in nuclear physics}{\faIcon{graduation-cap} University of Chinese Academy of Sciences (UCAS)}{Sept 2011 -- Aug 2012}{Beijing, China}
\cvevent{BSc, major in physics}{\faIcon{graduation-cap} Zhengzhou University (ZZU)}{Sept 2007 -- Aug 2011}{Zhengzhou, China}

\clearpage
\fi

\end{document}
